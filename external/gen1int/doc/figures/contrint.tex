\documentclass{article}

\usepackage[latin1]{inputenc}
\usepackage{tikz}

%\usepackage[textwidth=4cm,textheight=23cm]{geometry}

\setlength{\pdfpagewidth}{21.8cm}
\setlength{\pdfpageheight}{12.9cm}
%\setlength{\pdfvorigin}{25.4mm}

%\setlength{\textwidth}{16cm}
%\setlength{\textheight}{23cm}
%\setlength{\evensidemargin}{-3cm}
\setlength{\oddsidemargin}{-2.9cm}
\setlength{\topmargin}{-4.4cm}
\setlength{\headheight}{2cm}
\setlength{\headsep}{0.0cm}

\usetikzlibrary{shapes,arrows,shapes.geometric,calc}

\usepackage{amsmath}
\usepackage{amssymb}
\usepackage{amsthm}

\begin{document}
\pagestyle{empty}

% defines the block styles
\tikzstyle{group-block}=[rectangle, dashed, draw, sharp corners, very thin, node distance=0mm]
\tikzstyle{prim-block} = [color=blue, rectangle, draw, text width=13.5em, text centered, %
  sharp corners, minimum height=3em, node distance=15em, very thick, fill=blue!5]
\tikzstyle{basic-block} = [color=black, rectangle, draw, text width=9em, text centered, %
  sharp corners, minimum height=3em, node distance=15em, very thick, fill=black!5]
\tikzstyle{contr-block} = [color=red, rectangle, draw, text width=9em, text centered, %
  sharp corners, minimum height=3em, node distance=12em, very thick, fill=red!5]

\tikzstyle{arrow} = [draw, very thick, -latex']

\begin{tikzpicture}[scale=1, auto]
  % primitive integrals
  \node[group-block, color=blue, minimum width=215mm, minimum height=42mm] (prim-ints) at (79mm,-3mm){};
  \node at (-2mm,12mm) {\Large\color{blue}\textbf{Primitive integrals}};
  \node [prim-block] (f-carmom) at (0mm,0mm) %
    {$\left(\boldsymbol{\partial}_{\boldsymbol{M}}^{\boldsymbol{L}_{M}}%
        \boldsymbol{r}_{M}^{\boldsymbol{m}}\right)\boldsymbol{\partial_{r}^{n}}$};
  \node [prim-block, right of=f-carmom] (f-delta) %
    {$\left(\boldsymbol{\partial}_{\boldsymbol{M}}^{\boldsymbol{L}_{M}}\boldsymbol{r}_{M}^{\boldsymbol{m}}\right)%
      \left[\boldsymbol{\partial}_{\boldsymbol{C}}^{\boldsymbol{L}_{C}}\delta(\boldsymbol{r}_{C})\right]\boldsymbol{\partial_{r}^{n}}$};
  \node [prim-block, right of=f-delta] (f-nucpot) %
    {$\left(\boldsymbol{\partial}_{\boldsymbol{M}}^{\boldsymbol{L}_{M}}\boldsymbol{r}_{M}^{\boldsymbol{m}}\right)%
      \left[\boldsymbol{\partial}_{\boldsymbol{C}}^{\boldsymbol{L}_{C}}r_{C}^{-1}\right]\boldsymbol{\partial_{r}^{n}}$};
  \node [prim-block, right of=f-nucpot] (f-isdpot) %
    {$\left(\boldsymbol{\partial}_{\boldsymbol{M}}^{\boldsymbol{L}_{M}}\boldsymbol{r}_{M}^{\boldsymbol{m}}\right)%
      \left[\boldsymbol{\partial}_{\boldsymbol{C}}^{\boldsymbol{L}_{C}}r_{C}^{-2}\right]\boldsymbol{\partial_{r}^{n}}$};
  \node [prim-block, below of=f-carmom, node distance=3.9em] (f-gaupot) %
    {$\left(\boldsymbol{\partial}_{\boldsymbol{M}}^{\boldsymbol{L}_{M}}\boldsymbol{r}_{M}^{\boldsymbol{m}}\right)
        \left[\boldsymbol{\partial}_{\boldsymbol{C}}^{\boldsymbol{L}_{C}}\frac{\mathrm{erf}\left(\sqrt{\varrho}r_{C}\right)}{r_{C}}\right]%
        \boldsymbol{\partial_{r}^{n}}$};
  \node [prim-block, right of=f-gaupot] (f-dso) %
    {$\boldsymbol{\partial}_{\boldsymbol{C}_{1}}^{\boldsymbol{L}_{C_{1}}}%
      \boldsymbol{\partial}_{\boldsymbol{C}_{2}}^{\boldsymbol{L}_{C_{2}}}r_{C_{1}}^{-1}r_{C_{2}}^{-1}\boldsymbol{\partial_{r}^{n}}$};
  \node [prim-block, right of=f-dso] (f-ecp) {ECP};
  \node [prim-block, right of=f-ecp] (f-mcp1) {MCP\\(Version 1)};
  % quadrature
  \node [basic-block, below of=f-dso, node distance=25mm, xshift=27mm] (quadrature) {quadrature};
  % derivatives
  \node [basic-block, above of=prim-ints, xshift=-80mm, yshift=-20mm] (geo-part) {partial geometric\\derivatives};
  \node [basic-block, right of=geo-part] (mag-part-zero) {partial $\boldsymbol{B}$ and $\boldsymbol{J}$\\derivatives};
  % transformations
  \node [basic-block, right of=mag-part-zero] (hgto-to-sgto) {HGTOs to SGTOs};
  \node [basic-block, right of=hgto-to-sgto] (hgto-to-cgto) {HGTOs to CGTOs};
  % contracted integrals
  \node[group-block, color=red, minimum width=168mm, minimum height=28mm] (contr-ints) at (79mm,68mm){};
  \node at (24mm,77mm) {\Large\color{red}\textbf{Contracted integrals}};
  \node [contr-block, above of=geo-part, xshift=17mm, yshift=-9mm] (contr-lsgto) {contracted\\London SGTOs};
  \node [contr-block, right of=contr-lsgto] (contr-lcgto) {contracted\\London CGTOs};
  \node [contr-block, right of=contr-lcgto] (contr-sgto) {contracted SGTOs};
  \node [contr-block, right of=contr-sgto] (contr-cgto) {contracted CGTOs};
% draws arrows
  \path[arrow, color=black](quadrature)--(f-dso);
  \path[arrow, color=black](quadrature)--(f-ecp);
  \path[arrow, color=blue](prim-ints)--(contr-ints);
  \path[arrow, color=black](geo-part)--(contr-ints);
  \path[arrow, color=black](mag-part-zero)--(contr-lsgto);
  \path[arrow, color=black](mag-part-zero)--(contr-lcgto);
  \path[arrow, color=black](hgto-to-sgto)--(contr-lsgto);
  \path[arrow, color=black](hgto-to-sgto)--(contr-sgto);
  \path[arrow, color=black](hgto-to-cgto)--(contr-lcgto);
  \path[arrow, color=black](hgto-to-cgto)--(contr-cgto);
\end{tikzpicture}

\end{document}
